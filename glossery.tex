% AAAAAAAAAAAAAAAAAAAAAAAAAAAAAAAAAAAAAAAAAAAAAAAAAAAAAAAAAAAAAAAAAAAAAAAAAAAAAAAAAAAAAAAAAAAAAAAAAAA
\newglossaryentry{ANSSI}
{
    name=ANSSI,
    description={Agence nationale de la sécurité des systèmes d'information}
}

\newglossaryentry{API}
{
    name=API,
    description={Une API (application programming interface ou « interface de programmation d'application ») est une interface logicielle qui permet de « connecter » un logiciel ou un service à un autre logiciel ou service afin d'échanger des données et des fonctionnalités.}
}

\newglossaryentry{Apache}
{
    name=Apache,
    description={Le logiciel libre Apache HTTP Server est un serveur HTTP créé et maintenu au sein de la fondation Apache}
}

\newglossaryentry{Alma}
{
    name=Alma,
    description={AlmaLinux est une distribution Linux gratuite et open source, créée à l'origine par CloudLinux pour fournir un système d'exploitation d'entreprise de niveau production soutenu par la communauté et compatible binaire avec Red Hat Enterprise Linux}
}

\newglossaryentry{Ansible CMDB}
{
    name=Ansible CMDB,
    description={Ansible est une plate-forme logicielle libre pour la configuration et la gestion des ordinateurs. Il permet notamment de récupéré les informations des machines et d'en faire une synthèse dans une page web. C'est ce à quoi sert Ansible CMDB}
}
\newglossaryentry{Ansible}
{
    name=Ansible,
    description={Ansible est une plate-forme logicielle libre pour la configuration et la gestion des ordinateurs}
}
% BBBBBBBBBBBBBBBBBBBBBBBBBBBBBBBBBBBBBBBBBBBBBBBBBBBBBBBBBBBBBBBBBBBBBBBBBBBBBBBBBBBBBBBBBBBBBBBBBBB

\newglossaryentry{backups}
{
    name=Backups,
    description={En informatique, la sauvegarde ou backup est l'opération qui consiste à dupliquer et à mettre en sécurité les données contenues dans un système informatique}
}

\newglossaryentry{bastion SSH}
{
    name=Bastion SSH,
    description={Un bastion SSH est une brique d'infrastructure qui permet à une connexion SSH de “rebondir” avant d'atteindre sa cible}
}

\newglossaryentry{Bitwarden}
{
    name=Bitwarden,
    description={Bitwarden est un gestionnaire de mots de passe freemium et open source sous licence AGPL, qui permet de générer et de conserver des mots de passe de manière sécurisée. Ces éléments sont protégés par un seul et unique mot de passe appelé « mot de passe maître »}
}

\newglossaryentry{BigBlueButton}
{
    name=BigBlueButton,
    description={BigBlueButton est un système de visioconférence développé pour la formation à distance. Il permet le partage de la voix et de l'image vidéo, de présentations avec ou sans tableau blanc}
}

% CCCCCCCCCCCCCCCCCCCCCCCCCCCCCCCCCCCCCCCCCCCCCCCCCCCCCCCCCCCCCCCCCCCCCCCCCCCCCCCCCCCCCCCCCCCCCCCCCCC
\newglossaryentry{CVE}
{
    name=CVE,
    description={Common Vulnerabilities and Exposures. Dictionnaire des informations publiques relatives aux vulnérabilités de sécurité}
}

\newglossaryentry{CNI}
{
    name=CNI,
    description={CNI signifie Container Network Interface. De façon simplifier, les conteneurs dans un cluster Kubernetes ont besoin d'un réseau pour communiquer entre eux. C'est le rôle du CNI. Il existe de nombreux CNI disponibles pour Kubernetes}
}

\newglossaryentry{Calico}
{
    name=Calico,
    description={Calico est un nombreux CNI disponibles pour Kubernetes}
}

\newglossaryentry{cloud privé}
{
    name=Cloud privé,
    description={Le terme Cloud privé décrit un modèle de déploiement de Cloud à la demande avec lequel les services et l'infrastructure de Cloud Computing sont hébergés en privé sur l'intranet ou le Data Center de la société via des ressources propriétaires et ne sont pas partagés avec d'autres entreprises}
}

\newglossaryentry{cluster}
{
    name=Cluster,
    description={Groupe de serveurs et d’autres ressources qui agissent comme un système unique et permettent une haute disponibilité}
}

\newglossaryentry{Ceph}
{
    name=Ceph,
    description={Ceph est une solution libre de stockage distribué très populaire qui propose trois protocoles en un avec : Bloc, Fichiers et Objet. Les objectifs principaux de Ceph sont d'être complètement distribués sans point unique de défaillance, extensible jusqu'à l'exaoctet et librement disponible}
}
\newglossaryentry{CentOS}
{
    name=CentOS,
    description={CentOS est une distribution GNU/Linux destinée aux serveurs}
}
\newglossaryentry{Collabora}
{
    name=Collabora,
    description={Collabora est une suite bureautique en ligne basée sur LibreOffice avec des fonctions d'édition collaborative, qui prend en charge tous les principaux formats de documents, feuilles de calcul et fichiers de présentation, et fonctionne dans tous les navigateurs modernes.}
}
% DDDDDDDDDDDDDDDDDDDDDDDDDDDDDDDDDDDDDDDDDDDDDDDDDDDDDDDDDDDDDDDDDDDDDDDDDDDDDDDDDDDDDDDDDDDDDDDDDDD
\newglossaryentry{datacenters}
{
    name=Datacenter,
    description={Lieu où sont regroupés les équipements constituants d'un système d'information. Ce regroupement permet de faciliter la sécurisation, la gestion et la maintenance des équipements et des données stockées}
}
\newglossaryentry{DeepFence}
{
    name=DeepFence,
    description={Deepfence est une solution de prévention et de détection de sécurité essentielle pour les environnements cloud et conteneurs natifs}
}

\newglossaryentry{DHCP}
{
    name=DHCP,
    description={Dynamic Host Configuration Protocol est un protocole réseau dont le rôle est d’assurer la configuration automatique des paramètres IP d’une station ou d'une machine, notamment en lui attribuant automatiquement une adresse IP et un masque de sous-réseau}
}
% DDDDDDDDDDDDDDDDDDDDDDDDDDDDDDDDDDDDDDDDDDDDDDDDDDDDDDDDDDDDDDDDDDDDDDDDDDDDDDDDDDDDDDDDDDDDDDDDDDD
\newglossaryentry{DNS}
{
    name=DNS,
    description={Le Domain Name System ou DNS est un service informatique distribué utilisé qui traduit les noms de domaine Internet en adresse IP ou autres enregistrements}
}

\newglossaryentry{DPO}
{
    name=DPO,
    description={Délégué à la protection des données. En droit européen, le Délégué à la protection des données est la personne chargée de la protection des données personnelles au sein d'une organisation}
}

\newglossaryentry{Debian}
{
    name=Debian,
    description={Debian est un système d’exploitation Linux composée exclusivement de logiciels libres, développé par le Debian Project. Chaque version majeur de Debian possède une dénomination : Buster pour la version 10, Bulleyes pour la version 11 etc.}
}
\newglossaryentry{Docker-compose}
{
    name=Docker-compose,
    description={Docker-compose, un outil pour déployer plusieurs conteneurs en même temps. En gros, il faut retenir que Docker-compose permet de gérer un ensemble de conteneurs (services)}
}
\newglossaryentry{Docker}
{
    name=Docker,
    description={Docker est une plate-forme permettant de lancer certaines applications dans des conteneurs logiciels}
}

% EEEEEEEEEEEEEEEEEEEEEEEEEEEEEEEEEEEEEEEEEEEEEEEEEEEEEEEEEEEEEEEEEEEEEEEEEEEEEEEEEEEEEEEEEEEEEEEEEEE
\newglossaryentry{ESEO}
{
    name=ESEO,
    description={École Supérieure d'Électronique de l'Ouest. Dans le présent rapport, il est fait mention d'ESEO Angers}
}

% FFFFFFFFFFFFFFFFFFFFFFFFFFFFFFFFFFFFFFFFFFFFFFFFFFFFFFFFFFFFFFFFFFFFFFFFFFFFFFFFFFFFFFFFFFFFFFFFFFF
\newglossaryentry{Framemo}
{
    name=Framemo,
    description={Application de tableau blanc avec des colonnnes pour y déposer des postits}
}


% GGGGGGGGGGGGGGGGGGGGGGGGGGGGGGGGGGGGGGGGGGGGGGGGGGGGGGGGGGGGGGGGGGGGGGGGGGGGGGGGGGGGGGGGGGGGGGGGGGG
\newglossaryentry{GLPI}
{
    name=GLPI,
    description={GLPI est un logiciel libre de gestion des services informatiques et de gestion des services d'assistance. Cette solution libre est éditée en PHP et distribuée sous licence GPL. En tant que technologie libre, toute personne peut exécuter, modifier ou développer le code qui est libre}
}

% HHHHHHHHHHHHHHHHHHHHHHHHHHHHHHHHHHHHHHHHHHHHHHHHHHHHHHHHHHHHHHHHHHHHHHHHHHHHHHHHHHHHHHHHHHHHHHHHHHH
\newglossaryentry{Haproxy}
{
    name=Haproxy,
    description={HAProxy est un logiciel gratuit et open source qui fournit un équilibreur de charge haute disponibilité et un proxy inverse pour les applications TCP et HTTP qui répartissent les requêtes sur plusieurs serveurs}
}

\newglossaryentry{HTTP}
{
    name=HTTP,
    description={L’Hypertext Transfer Protocol, généralement abrégé HTTP, littéralement « protocole de transfert hypertexte », est un protocole de communication client-serveur}
}

\newglossaryentry{HTTPS}
{
    name=HTTPS,
    description={L'HyperText Transfer Protocol Secure est la combinaison du HTTP avec une couche de chiffrement comme SSL ou TLS. HTTPS permet au visiteur de vérifier l'identité du site web auquel il accède, grâce à un certificat d'authentification émis par une autorité tierce, réputée fiable}
}

\newglossaryentry{HTML}
{
    name=HTML,
    description={Le HyperText Markup Language, généralement abrégé HTML ou, dans sa dernière version, HTML5, est le langage de balisage conçu pour représenter les pages web}
}
% IIIIIIIIIIIIIIIIIIIIIIIIIIIIIIIIIIIIIIIIIIIIIIIIIIIIIIIIIIIIIIIIIIIIIIIIIIIIIIIIIIIIIIIIIIIIIIIIIII
\newglossaryentry{IPsec}
{
    name=IPsec,
    description={Internet Protocol Security .Regroupe un ensemble de protocoles, qui utilisent des algorithmes destinés à transporter des données sur un réseau de façon sécurisée}
}

\newglossaryentry{IP}
{
    name=IP,
    description={Une adresse IP est un numéro d'identification qui est attribué de façon permanente ou provisoire à chaque périphérique relié à un réseau informatique qui utilise l'Internet Protocol. L'adresse IP est à la base du système d'acheminement des paquets de données sur Internet}
}

% JJJJJJJJJJJJJJJJJJJJJJJJJJJJJJJJJJJJJJJJJJJJJJJJJJJJJJJJJJJJJJJJJJJJJJJJJJJJJJJJJJJJJJJJJJJJJJJJJJJ

% KKKKKKKKKKKKKKKKKKKKKKKKKKKKKKKKKKKKKKKKKKKKKKKKKKKKKKKKKKKKKKKKKKKKKKKKKKKKKKKKKKKKKKKKKKKKKKKKKKK
\newglossaryentry{Kubernetes}
{
    name=Kubernetes,
    description={Kubernetes est un système open source qui vise à fournir une « plate-forme permettant d'automatiser le déploiement, la montée en charge et la mise en œuvre de conteneurs d'application sur des clusters de serveurs »}
}

\newglossaryentry{Kubespray}
{
    name=Kubespray,
    description={Utilitaire permettant de créer un cluster Kubernetes via Ansible}
}

% LLLLLLLLLLLLLLLLLLLLLLLLLLLLLLLLLLLLLLLLLLLLLLLLLLLLLLLLLLLLLLLLLLLLLLLLLLLLLLLLLLLLLLLLLLLLLLLLLLL
\newglossaryentry{LAMP}
{
    name=LAMP,
    description={LAMP est un acronyme désignant un ensemble de logiciels libres permettant de construire des serveurs de sites web. Linux Apache Mysql Mariadb}
}

\newglossaryentry{LDAP}
{
    name=LDAP,
    description={Lightweight Directory Access Protocol est un protocole permettant l'interrogation et la modification des services d'annuaire électronique.}
}

\newglossaryentry{Linux}
{
    name=Linux,
    description={Linux ou GNU/Linux est une famille de systèmes d'exploitation open source de type Unix fondé sur le noyau Linux, créé en 1991 par Linus Torvalds}
}
\newglossaryentry{LimeSurvey}
{
    name=LimeSurvey,
    description={LimeSurvey est un logiciel d'enquête statistique, de sondage, et de création de formulaires en ligne}
}
% MMMMMMMMMMMMMMMMMMMMMMMMMMMMMMMMMMMMMMMMMMMMMMMMMMMMMMMMMMMMMMMMMMMMMMMMMMMMMMMMMMMMMMMMMMMMMMMMMMM
\newglossaryentry{Matomo}
{
    name=Matomo,
    description={Matomo, anciennement Piwik jusqu’au début de 2018, est un logiciel libre et open source de mesure de statistiques web, successeur de PhpMyVisites et conçu pour être une alternative libre à Google Analytics}
}
\newglossaryentry{Master/Slave}
{
    name=Master/Slave,
    description={Le principe de Master/Slave consiste à avoir 2 logiciels ou base de données interdépendant. Dans l'idée, une modification apportée au Master sera par la suite automatiquement appliquée au Slave.}
}

\newglossaryentry{Mariadb}
{
    name=Mariadb,
    description={MariaDB est un système de gestion de base de données édité sous licence GPL. Il s'agit d'un embranchement communautaire de MySQL : la gouvernance du projet est assurée par la fondation MariaDB, et sa maintenance par la société Monty Program AB, créateur du projet}
}


\newglossaryentry{Mattermost}
{
    name=Mattermost,
    description={Mattermost est un logiciel et un service de messagerie instantanée libre auto-hébergeable. Il est conçu comme un chat interne pour les organisations et les entreprises, et il est présenté comme une alternative à Slack et Microsoft Teams}
}
\newglossaryentry{Moodle}
{
    name=Moodle,
    description={Moodle est une plate-forme d'apprentissage en ligne libre distribuée sous la Licence publique générale GNU écrite en PHP. Développée à partir de principes pédagogiques, elle permet de créer des communautés s'instruisant autour de contenus et d'activités}
}
% NNNNNNNNNNNNNNNNNNNNNNNNNNNNNNNNNNNNNNNNNNNNNNNNNNNNNNNNNNNNNNNNNNNNNNNNNNNNNNNNNNNNNNNNNNNNNNNNNNN

\newglossaryentry{Nginx}
{
    name=Nginx,
    description={Nginx est un logiciel libre de serveur Web ainsi qu'un proxy inverse}
}
\newglossaryentry{Nextcloud}
{
    name=Nextcloud,
    description={Nextcloud est un logiciel libre de site d'hébergement de fichiers et une plate-forme de collaboration}
}
\newglossaryentry{NocoDB}
{
    name=NocoDB,
    description={Outils permettant le traitement en feuille de calcul de bases de données SQL}
}

% OOOOOOOOOOOOOOOOOOOOOOOOOOOOOOOOOOOOOOOOOOOOOOOOOOOOOOOOOOOOOOOOOOOOOOOOOOOOOOOOOOOOOOOOOOOOOOOOOOO
\newglossaryentry{openvas}
{
    name=Openvas,
    description={OpenVAS, est un fork sous licence GNU GPL du scanner de vulnérabilité Nessus dont le but est de permettre un développement libre de l’outil qui est maintenant sous licence propriétaire}
}

\newglossaryentry{OpenSCAP}
{
    name=OpenSCAP,
    description={OpenSCAP fournit plusieurs outils pour aider les administrateurs et les auditeurs à évaluer, mesurer et appliquer les lignes de base de sécurité}
}

\newglossaryentry{OSD}
{
    name=OSD,
    description={Object Storage Deamon. Dans ceph, un OSD est le deamon responsable du suivi des stockages sous forme d'objet}
}

\newglossaryentry{OCSInventory}
{
    name=OCSInventory,
    description={OCS Inventory NG soit Open Computer and Software Inventory est une application permettant de réaliser un inventaire sur la configuration matérielle des machines du réseau, sur les logiciels qui y sont installés et de visualiser ces informations grâce à une interface web}
}


% PPPPPPPPPPPPPPPPPPPPPPPPPPPPPPPPPPPPPPPPPPPPPPPPPPPPPPPPPPPPPPPPPPPPPPPPPPPPPPPPPPPPPPPPPPPPPPPPPPP

\newglossaryentry{plugin}
{
    name=Plugin,
    description={En informatique, un plugin ou plug-in, aussi nommé module d'extension, module externe, greffon, plugiciel, ainsi qu'add-in ou add-on en France, est un logiciel conçu pour être greffé à un autre logiciel à travers une interface prévue à cet effet, et apporter à ce dernier de nouvelles fonctionnalités}
}

\newglossaryentry{passhport}
{
    name=Passhport,
    description={Solution de gestion et de sécurisation des accès SSH. PaSSHport est un bastion SSH protégeant les accès sécurisés des systèmes d'information}
}

\newglossaryentry{playbooks}
{
    name=Playbook,
    description={Les Playbooks Ansible offrent un système de gestion de configuration et de déploiement multi-machine reproductible, réutilisable et simple, bien adapté au déploiement d'applications complexes}
}

\newglossaryentry{plate-forme web}
{
    name=plate-forme web,
    description={Une plate-forme web est un ensemble de services web. Le contenu des plate-formes web provient en grande partie des utilisateurs qui peuvent diffuser et partager des contenus de nature textuelle ou multimédia}
}

\newglossaryentry{Peertube}
{
    name=Peertube,
    description={PeerTube est un logiciel libre d'hébergement de vidéo décentralisé permettant la diffusion en pair à pair, et un média social sur lequel les utilisateurs peuvent envoyer, regarder, commenter, évaluer et partager des vidéos en streaming}
}

\newglossaryentry{POC}
{
    name=POC,
    description={Proof of Concept ou Preuve de concept. Démonstration de faisabilité, c'est à dire une réalisation expérimentale concrète et préliminaire, courte ou incomplète, illustrant une certaine méthode ou idée afin d'en démontrer ou pas la faisabilité}
}

\newglossaryentry{PSSI}
{
    name=PSSI,
    description={Politique de sécurité du système d'information. Plan d'actions définies pour maintenir un certain niveau de sécurité. Elle reflète la vision stratégique de la direction de l'organisme en matière de sécurité des systèmes d'information}
}

\newglossaryentry{PFE}
{
    name=PFE,
    description={Projet de fin d'étude. Désigne à l'ESEO à un projet d'une durée d'un semestre pouvant être effectué en entreprise ou non}
}

\newglossaryentry{Proxmox}
{
    name=Proxmox,
    description={Proxmox Virtual Environment est une solution de virtualisation libre basée sur l'hyperviseur Linux KVM, et offre aussi une solution de containers avec LXC}
}

\newglossaryentry{proxy}
{
    name=Proxy,
    description={Un proxy est un composant logiciel informatique qui joue le rôle d'intermédiaire en se plaçant entre deux hôtes pour faciliter ou surveiller leurs échanges}
}
\newglossaryentry{Passhport}
{
    name=Passhport,
    description={Passhport est une solution de gestion et de sécurisation des accès SSH}
}
\newglossaryentry{PHP}
{
    name=PHP,
    description={PHP: Hypertext Preprocessor est un langage de programmation principalement utilisé pour produire des pages Web dynamiques via un serveur HTTP, mais pouvant également fonctionner comme n'importe quel langage interprété de façon locale. PHP est un langage impératif orienté objet}
}


% QQQQQQQQQQQQQQQQQQQQQQQQQQQQQQQQQQQQQQQQQQQQQQQQQQQQQQQQQQQQQQQQQQQQQQQQQQQQQQQQQQQQQQQQQQQQQQQQQQQ

% RRRRRRRRRRRRRRRRRRRRRRRRRRRRRRRRRRRRRRRRRRRRRRRRRRRRRRRRRRRRRRRRRRRRRRRRRRRRRRRRRRRRRRRRRRRRRRRRRRR
\newglossaryentry{Rudder}
{
    name=Rudder,
    description={Rudder est un logiciel libre de configuration automatique de serveurs. Il se veut simple d'utilisation, orienté web et applique un raisonnement dirigé par les rôles. Il s'appuie sur des agents légers installés localement sur chaque machine gérée}
}

\newglossaryentry{Rocky}
{
    name=Rocky Linux,
    description={Rocky Linux est une distribution Linux basée sur le code source du système d'exploitation Red Hat Enterprise Linux.}
}

\newglossaryentry{RGPD}
{
    name=RGPD,
    description={Règlement Générale sur la Protection des Données}
}
\newglossaryentry{Redmine}
{
    name=Redmine,
    description={Redmine est une application web libre de gestion de projets, développée en Ruby sur la base du framework Ruby on Rails}
}

% SSSSSSSSSSSSSSSSSSSSSSSSSSSSSSSSSSSSSSSSSSSSSSSSSSSSSSSSSSSSSSSSSSSSSSSSSSSSSSSSSSSSSSSSSSSSSSSSSSS
\newglossaryentry{sysadmin}
{
    name=Sysadmin,
    description={Le pôle sysadmin est en charge du développement, du suivi et de la maintenance du SI de l'entreprise. C'est l'équivalent d'administrateurs systèmes}
}

\newglossaryentry{SQL}
{
    name=SQL,
    description={SQL est un langage informatique normalisé servant à exploiter des bases de données relationnelles. La partie langage de manipulation des données de SQL permet de rechercher, d'ajouter, de modifier ou de supprimer des données dans les bases de données relationnelles}
}

\newglossaryentry{SCOP}
{
    name=SCOP,
    description={Société coopérative et participative}
}

\newglossaryentry{SI}
{
    name=SI,
    description={Système d'information. Ensemble organisé de ressources qui permet de collecter, stocker, traiter et distribuer de l'information, en général grâce à un réseau d'ordinateurs}
}

\newglossaryentry{SSH}
{
    name=SSH,
    description={Protocole de communication sécurisée basé sur un échange de clé privé et publique}
}

\newglossaryentry{SSL}
{
    name=SSL,
    description={Le protocole SSL (Secure Sockets Layer) était le protocole cryptographique le plus largement utilisé pour assurer la sécurité des communications sur Internet}
}
\newglossaryentry{SNMP}
{
    name=SNMP,
    description={Simple Network Management Protocol, en français « protocole simple de gestion de réseau », est un protocole de communication qui permet aux administrateurs réseau de gérer les équipements du réseau, de superviser et de diagnostiquer des problèmes réseaux et matériels à distance}
}

% TTTTTTTTTTTTTTTTTTTTTTTTTTTTTTTTTTTTTTTTTTTTTTTTTTTTTTTTTTTTTTTTTTTTTTTTTTTTTTTTTTTTTTTTTTTTTTTTTTT
\newglossaryentry{TLS}
{
    name=TLS,
    description={Transport Layer Security ou Sécurité de la couche de transport est un protocole de sécurisation des échanges par réseau informatique, notamment par Internet}
}
\newglossaryentry{Traefik}
{
    name=Traefik,
    description={Traefik est donc un reverse-proxy et un load-balancer fait pour déployer principalement des conteneurs}
}
\newglossaryentry{Terraform}
{
    name=Terraform,
    description={Terraform est un environnement logiciel d'« infrastructure as code » publié en open-source par la société HashiCorp. Cet outil permet d'automatiser la construction des ressources d'une infrastructure de centre de données comme un réseau, des machines virtuelles, un groupe de sécurité ou une base de données}
}

% UUUUUUUUUUUUUUUUUUUUUUUUUUUUUUUUUUUUUUUUUUUUUUUUUUUUUUUUUUUUUUUUUUUUUUUUUUUUUUUUUUUUUUUUUUUUUUUUUUU

% VVVVVVVVVVVVVVVVVVVVVVVVVVVVVVVVVVVVVVVVVVVVVVVVVVVVVVVVVVVVVVVVVVVVVVVVVVVVVVVVVVVVVVVVVVVVVVVVVVV
\newglossaryentry{vip}
{
    name=vip,
    description={Une adresse IP virtuelle est une adresse IP non connectée à un ordinateur ou une carte réseau spécifiques. Les paquets entrants sont envoyés à l'adresse IP virtuelle, mais en réalité ils circulent tous via des interfaces réseau réelles}
}

\newglossaryentry{vhosts}
{
    name=Vhosts,
    description={En informatique, l'hébergement virtuel est une méthode que les serveurs tels que serveurs Web utilisent pour accueillir plus d'un nom de domaine sur le même ordinateur, parfois sur la même adresse IP, tout en maintenant une gestion séparée de chacun de ces noms}
}

% WWWWWWWWWWWWWWWWWWWWWWWWWWWWWWWWWWWWWWWWWWWWWWWWWWWWWWWWWWWWWWWWWWWWWWWWWWWWWWWWWWWWWWWWWWWWWWWWWWW
\newglossaryentry{Wifi}
{
    name=Wifi,
    description={Le terme Wifi correspond au protocole IEEE 802.11 est un ensemble de normes concernant les réseaux sans fil locaux}
}

\newglossaryentry{webinaires}
{
    name=Webinaire,
    description={Webinaire est un mot-valise associant les mots Web et séminaire, créé pour désigner toutes les formes de réunions interactives de type séminaire faites par Internet généralement dans un but de travail collaboratif ou de transmission d'informations pour une audience plus ou moins importante en nombre}
}


\newglossaryentry{Windows}
{
    name=Windows,
    description={Windows est au départ une interface graphique unifiée produite par Microsoft, qui est devenue ensuite une gamme de systèmes d’exploitation à part entière, principalement destinés aux ordinateurs compatibles PC}
}
\newglossaryentry{Wazuh}
{
    name=Wazuh,
    description={Wazuh est une plate-forme open source utilisée pour la prévention, la détection et la réponse aux menaces}
}

% XXXXXXXXXXXXXXXXXXXXXXXXXXXXXXXXXXXXXXXXXXXXXXXXXXXXXXXXXXXXXXXXXXXXXXXXXXXXXXXXXXXXXXXXXXXXXXXXXXX

% YYYYYYYYYYYYYYYYYYYYYYYYYYYYYYYYYYYYYYYYYYYYYYYYYYYYYYYYYYYYYYYYYYYYYYYYYYYYYYYYYYYYYYYYYYYYYYYYYYY
\newglossaryentry{YesWeHack}
{
    name=YesWeHack,
    description={YesWeHack est une plate-forme permettant la mise en contact entre des entreprises et des hackers éthiques}
}

% ZZZZZZZZZZZZZZZZZZZZZZZZZZZZZZZZZZZZZZZZZZZZZZZZZZZZZZZZZZZZZZZZZZZZZZZZZZZZZZZZZZZZZZZZZZZZZZZZZZZ
\newglossaryentry{Zammad}
{
    name=Zammad,
    description={Zammad est un service d'assistance gratuit ou un système de suivi des problèmes}
}

































