% AAAAAAAAAAAAAAAAAAAAAAAAAAAAAAAAAAAAAAAAAAAAAAAAAAAAAAAAAAAAAAAAAAAAAAAAAAAAAAAAAAAAAAAAAAAAAAAAAAA
\newglossaryentry{ANSSI}
{
    name=ANSSI,
    description={Agence Nationale de la Sécurité des Systèmes d'Information. Organisme Français expert dans le domaine de la sécurité}
}

\newglossaryentry{API}
{
    name=API,
    description={Application Programming interface. Interface logicielle qui permet de « connecter » un logiciel ou un service à un autre logiciel ou service afin d'échanger des données et des fonctionnalités.}
}

\newglossaryentry{Apache}
{
    name=Apache,
    description={Serveur HTTP libre créé et maintenu au sein de la fondation Apache}
}

\newglossaryentry{Alma}
{
    name=Alma,
    description={Distrubtion Linux. Système d'exploitation d'entreprise de niveau production soutenu par la communauté et compatible binaire avec Red Hat Enterprise Linux}
}

\newglossaryentry{Ansible CMDB}
{
    name=Ansible CMDB,
    description={Ansible est une plateforme logicielle libre pour la configuration et la gestion des ordinateurs. Il permet notamment de récupéré les informations des machines et d'en faire une synthèse dans une page web}
}
\newglossaryentry{Ansible}
{
    name=Ansible,
    description={Ansible est une plateforme logicielle libre pour la configuration et la gestion des ordinateurs}
}
\newglossaryentry{apt}
{
    name=apt,
    description={Advanced Packaging Tool. Gestionnaire de paquets utilisé par Debian et ses dérivés}
}

% BBBBBBBBBBBBBBBBBBBBBBBBBBBBBBBBBBBBBBBBBBBBBBBBBBBBBBBBBBBBBBBBBBBBBBBBBBBBBBBBBBBBBBBBBBBBBBBBBBB

\newglossaryentry{backups}
{
    name=Backups,
    description={En informatique, la sauvegarde ou backup est l'opération qui consiste à dupliquer et à mettre en sécurité les données contenues dans un système informatique}
}

\newglossaryentry{bastion SSH}
{
    name=Bastion SSH,
    description={Un bastion SSH est une brique d'infrastructure qui permet à une connexion SSH de “rebondir” avant d'atteindre sa cible}
}

\newglossaryentry{Bitwarden}
{
    name=Bitwarden,
    description={Bitwarden est un gestionnaire de mots de passe freemium et open source sous licence AGPL, qui permet de générer et de conserver des mots de passe de manière sécurisée. Ces éléments sont protégés par un seul et unique mot de passe appelé « mot de passe maître »}
}

\newglossaryentry{BigBlueButton}
{
    name=BigBlueButton,
    description={BigBlueButton est un système de visioconférence développé pour la formation à distance. Il permet le partage de la voix et de l'image vidéo, de présentations avec ou sans tableau blanc}
}

% CCCCCCCCCCCCCCCCCCCCCCCCCCCCCCCCCCCCCCCCCCCCCCCCCCCCCCCCCCCCCCCCCCCCCCCCCCCCCCCCCCCCCCCCCCCCCCCCCCC
\newglossaryentry{CVE}
{
    name=CVE,
    description={Common Vulnerabilities and Exposures. Dictionnaire des informations publiques relatives aux vulnérabilités de sécurité}
}

\newglossaryentry{CNI}
{
    name=CNI,
    description={Container Network Interface. De façon simplifier, les conteneurs dans un cluster Kubernetes ont besoin d'un réseau pour communiquer entre eux. C'est le rôle du CNI. Il existe de nombreux CNI disponibles pour Kubernetes}
}

\newglossaryentry{Calico}
{
    name=Calico,
    description={Calico est un des nombreux CNI disponibles pour Kubernetes}
}

\newglossaryentry{CLI}
{
    name=CLI,
    description={Command Line Interface. Interface en ligne de commande}
}

\newglossaryentry{cloud privé}
{
    name=Cloud privé,
    description={Le terme Cloud privé décrit un modèle de déploiement de Cloud à la demande avec lequel les services et l'infrastructure de Cloud Computing sont hébergés en privé sur l'intranet ou le Data Center de la société via des ressources propriétaires et ne sont pas partagés avec d'autres entreprises}
}

\newglossaryentry{cluster}
{
    name=cluster,
    description={Groupement de serveurs et d’autres ressources qui agissent comme un système unique et permettent une haute disponibilité}
}

\newglossaryentry{Ceph}
{
    name=Ceph,
    description={Ceph est une solution libre de stockage distribué très populaire qui propose trois protocoles en un avec : Bloc, Fichiers et Objet. Les objectifs principaux de Ceph sont d'être complètement distribués sans point unique de défaillance, extensible jusqu'à l'exaoctet et librement disponible}
}
\newglossaryentry{CentOS}
{
    name=CentOS,
    description={Distribution Linux. CentOS est une distribution GNU/Linux destinée aux serveurs}
}
\newglossaryentry{Collabora}
{
    name=Collabora,
    description={Collabora est une suite bureautique en ligne basée sur LibreOffice avec des fonctions d'édition collaborative, qui prend en charge tous les principaux formats de documents, feuilles de calcul et fichiers de présentation, et fonctionne dans tous les navigateurs modernes.}
}
% DDDDDDDDDDDDDDDDDDDDDDDDDDDDDDDDDDDDDDDDDDDDDDDDDDDDDDDDDDDDDDDDDDDDDDDDDDDDDDDDDDDDDDDDDDDDDDDDDDD
\newglossaryentry{DevOps}
{
    name=DevOps,
    description={Mouvement en ingénierie informatique et une pratique technique visant à l'unification du développement logiciel et de l'administration des infrastructures informatiques, notamment l'administration système}
}

\newglossaryentry{DNI}
{
    name=DNI,
    description={Détecteur de Niveau Intelligent}
}

\newglossaryentry{datacenters}
{
    name=Datacenter,
    description={Lieu où sont regroupés les équipements constituants d'un système d'information. Ce regroupement permet de faciliter la sécurisation, la gestion et la maintenance des équipements et des données stockées}
}
\newglossaryentry{DeepFence}
{
    name=DeepFence,
    description={Deepfence est une solution de prévention et de détection de sécurité essentielle pour les environnements cloud et conteneurs natifs}
}

\newglossaryentry{DHCP}
{
    name=DHCP,
    description={Dynamic Host Configuration Protocol. Protocole réseau dont le rôle est d’assurer la configuration automatique des paramètres IP d’une station ou d'une machine, notamment en lui attribuant automatiquement une adresse IP et un masque de sous-réseau}
}
% DDDDDDDDDDDDDDDDDDDDDDDDDDDDDDDDDDDDDDDDDDDDDDDDDDDDDDDDDDDDDDDDDDDDDDDDDDDDDDDDDDDDDDDDDDDDDDDDDDD
\newglossaryentry{DNS}
{
    name=DNS,
    description={Domain Name System. Service informatique distribué utilisé qui traduit les noms de domaine Internet en adresse IP ou autres enregistrements}
}
\newglossaryentry{DPO}
{
    name=DPO,
    description={Délégué à la Protection des Données. En droit européen, le Délégué à la protection des données est la personne chargée de la protection des données personnelles au sein d'une organisation}
}
\newglossaryentry{Debian}
{
    name=Debian,
    description={Debian est un système d’exploitation Linux composée exclusivement de logiciels libres, développé par le Debian Project. Chaque version majeur de Debian possède une dénomination : Buster pour la version 10, Bulleyes pour la version 11 etc.}
}
\newglossaryentry{Docker-compose}
{
    name=Docker-compose,
    description={Docker-compose, un outil pour déployer plusieurs conteneurs en même temps. En gros, il faut retenir que Docker-compose permet de gérer un ensemble de conteneurs (services)}
}
\newglossaryentry{Docker}
{
    name=Docker,
    description={Docker est une plateforme permettant de lancer certaines applications dans des conteneurs logiciels}
}

% EEEEEEEEEEEEEEEEEEEEEEEEEEEEEEEEEEEEEEEEEEEEEEEEEEEEEEEEEEEEEEEEEEEEEEEEEEEEEEEEEEEEEEEEEEEEEEEEEEE
\newglossaryentry{ESEO}
{
    name=ESEO,
    description={École Supérieure d'Électronique de l'Ouest. Dans le présent rapport, il est fait mention d'ESEO Angers}
}

% FFFFFFFFFFFFFFFFFFFFFFFFFFFFFFFFFFFFFFFFFFFFFFFFFFFFFFFFFFFFFFFFFFFFFFFFFFFFFFFFFFFFFFFFFFFFFFFFFFF
\newglossaryentry{FileSystem}
{
    name=FileSystem,
    description={Système de fichiers. Organisation des fichiers au sein d'un volume physique ou logique, qui peut être de différents types, et qui a également une racine mais peut en avoir plusieurs}
}

\newglossaryentry{Framemo}
{
    name=Framemo,
    description={Application de tableau blanc avec des colonnes pour y déposer des postits}
}
% GGGGGGGGGGGGGGGGGGGGGGGGGGGGGGGGGGGGGGGGGGGGGGGGGGGGGGGGGGGGGGGGGGGGGGGGGGGGGGGGGGGGGGGGGGGGGGGGGGG
\newglossaryentry{GLPI}
{
    name=GLPI,
    description={Gestionnaire Libre de Parc Informatique. Logiciel libre de gestion des services informatiques et de gestion des services d'assistance. Cette solution libre est éditée en PHP et distribuée sous licence GPL. En tant que technologie libre, toute personne peut exécuter, modifier ou développer le code qui est libre}
}
\newglossaryentry{GitLab}
{
    name=GitLab,
    description={GitLab est un logiciel libre de forge basé sur git proposant les fonctionnalités de wiki, un système de suivi des bugs, l’intégration continue et la livraison continue}
}

% HHHHHHHHHHHHHHHHHHHHHHHHHHHHHHHHHHHHHHHHHHHHHHHHHHHHHHHHHHHHHHHHHHHHHHHHHHHHHHHHHHHHHHHHHHHHHHHHHHH
\newglossaryentry{HDD}
{
    name=HDD,
    description={Hard Disk Drive. Mémoire de masse à disque tournant magnétique utilisée principalement dans les ordinateurs}
}
\newglossaryentry{Helm}
{
    name=Helm,
    description={Helm est un outil d'empaquetage open source qui vous aide à installer et à gérer le cycle de vie d'applications Kubernetes}
}
\newglossaryentry{Haproxy}
{
    name=Haproxy,
    description={High Availability Proxy. Logiciel gratuit et open source qui fournit un équilibreur de charge haute disponibilité et un proxy inverse pour les applications TCP et HTTP qui répartissent les requêtes sur plusieurs serveurs}
}

\newglossaryentry{HTTP}
{
    name=HTTP,
    description={Hypertext Transfer Protocol. Protocole de communication client-serveur}
}

\newglossaryentry{HTTPS}
{
    name=HTTPS,
    description={HyperText Transfer Protocol Secure. Combinaison du HTTP avec une couche de chiffrement comme SSL ou TLS. HTTPS permet au visiteur de vérifier l'identité du site web auquel il accède, grâce à un certificat d'authentification émis par une autorité tierce, réputée fiable}
}

\newglossaryentry{HTML}
{
    name=HTML,
    description={HyperText Markup Language. Langage de balisage conçu pour représenter les pages web}
}
% IIIIIIIIIIIIIIIIIIIIIIIIIIIIIIIIIIIIIIIIIIIIIIIIIIIIIIIIIIIIIIIIIIIIIIIIIIIIIIIIIIIIIIIIIIIIIIIIIII
\newglossaryentry{IPsec}
{
    name=IPsec,
    description={Internet Protocol Security. Regroupe un ensemble de protocoles, qui utilisent des algorithmes destinés à transporter des données sur un réseau de façon sécurisée}
}

\newglossaryentry{IP}
{
    name=IP,
    description={Internal Protocole. Numéro d'identification qui est attribué de façon permanente ou provisoire à chaque périphérique relié à un réseau informatique qui utilise l'Internet Protocol. L'adresse IP est à la base du système d'acheminement des paquets de données sur Internet}
}

\newglossaryentry{IaC}
{
    name=IaC,
    description={Infrastructure as Code. Consiste à gérer et à approvisionner une infrastructure à l'aide de code}
}
\newglossaryentry{ISO}
{
    name=ISO,
    description={Optical disc image. Image de disque qui contient tout ce qui serait écrit sur un disque optique, secteur de disque par secteur de disque, y compris le système de fichiers du disque optique}
}


% JJJJJJJJJJJJJJJJJJJJJJJJJJJJJJJJJJJJJJJJJJJJJJJJJJJJJJJJJJJJJJJJJJJJJJJJJJJJJJJJJJJJJJJJJJJJJJJJJJJ
\newglossaryentry{Jinja}
{
    name=Jinja,
    description={Jinja est un moteur de templates pour le langage de programmation Python. Il aurait inspiré Twig, moteur de template PHP}
}


% KKKKKKKKKKKKKKKKKKKKKKKKKKKKKKKKKKKKKKKKKKKKKKKKKKKKKKKKKKKKKKKKKKKKKKKKKKKKKKKKKKKKKKKKKKKKKKKKKKK
\newglossaryentry{Kubernetes}
{
    name=Kubernetes,
    description={Kubernetes est un système open source qui vise à fournir une « plateforme permettant d'automatiser le déploiement, la montée en charge et la mise en œuvre de conteneurs d'application sur des clusters de serveurs »}
}

\newglossaryentry{Kubespray}
{
    name=Kubespray,
    description={Utilitaire permettant de créer un cluster Kubernetes via Ansible}
}

% LLLLLLLLLLLLLLLLLLLLLLLLLLLLLLLLLLLLLLLLLLLLLLLLLLLLLLLLLLLLLLLLLLLLLLLLLLLLLLLLLLLLLLLLLLLLLLLLLLL
\newglossaryentry{LAMP}
{
    name=LAMP,
    description={Linux Apache MySQL Php. LAMP est un acronyme désignant un ensemble de logiciels libres permettant de construire des serveurs de sites web. Linux Apache Mysql Mariadb}
}

\newglossaryentry{LDAP}
{
    name=LDAP,
    description={Lightweight Directory Access Protocol. Protocole permettant l'interrogation et la modification des services d'annuaire électronique.}
}

\newglossaryentry{Linux}
{
    name=Linux,
    description={Famille de systèmes d'exploitation open source de type Unix fondé sur le noyau Linux, créé en 1991 par Linus Torvalds}
}
\newglossaryentry{LimeSurvey}
{
    name=LimeSurvey,
    description={Logiciel d'enquête statistique, de sondage, et de création de formulaires en ligne}
}
% MMMMMMMMMMMMMMMMMMMMMMMMMMMMMMMMMMMMMMMMMMMMMMMMMMMMMMMMMMMMMMMMMMMMMMMMMMMMMMMMMMMMMMMMMMMMMMMMMMM
\newglossaryentry{Microsoft Teams}
{
    name=Microsoft Teams,
    description={Microsoft Teams est une application de communication collaborative propriétaire en mode SaaS officiellement lancée par Microsoft en novembre 2016}
}
\newglossaryentry{Matomo}
{
    name=Matomo,
    description={Logiciel libre et open source de mesure de statistiques web, successeur de PhpMyVisites et conçu pour être une alternative libre à Google Analytics}
}
\newglossaryentry{Markdown}
{
    name=Markdown,
    description={Langage de balisage léger. Il a été créé dans le but d'offrir une syntaxe facile à lire et à écrire. Un document balisé par Markdown peut être lu en l'état sans donner l’impression d'avoir été balisé ou formaté par des instructions particulières.}
}
\newglossaryentry{Master/Slave}
{
    name=Master/Slave,
    description={Le principe de Master/Slave consiste à avoir deyx logiciels ou base de données interdépendant. Dans l'idée, une modification apportée au Master sera par la suite automatiquement appliquée au Slave.}
}

\newglossaryentry{Mariadb}
{
    name=Mariadb,
    description={Système de gestion de base de données édité sous licence GPL. Il s'agit d'un embranchement communautaire de MySQL : la gouvernance du projet est assurée par la fondation MariaDB, et sa maintenance par la société Monty Program AB, créateur du projet}
}


\newglossaryentry{Mattermost}
{
    name=Mattermost,
    description={Logiciel et un service de messagerie instantanée libre auto-hébergeable. Il est conçu comme un chat interne pour les organisations et les entreprises, et il est présenté comme une alternative à Slack et Microsoft Teams}
}
\newglossaryentry{Moodle}
{
    name=Moodle,
    description={Plateforme d'apprentissage en ligne libre distribuée sous la Licence publique générale GNU écrite en PHP. Développée à partir de principes pédagogiques, elle permet de créer des communautés s'instruisant autour de contenus et d'activités}
}
% NNNNNNNNNNNNNNNNNNNNNNNNNNNNNNNNNNNNNNNNNNNNNNNNNNNNNNNNNNNNNNNNNNNNNNNNNNNNNNNNNNNNNNNNNNNNNNNNNNN
\newglossaryentry{NFS}
{
    name=NFS,
    description={Network File System. Protocole permettant d'accéder à des fichiers via un réseau}
}
\newglossaryentry{Nginx}
{
    name=Nginx,
    description={Logiciel libre de serveur Web ainsi qu'un proxy inverse}
}
\newglossaryentry{Nextcloud}
{
    name=Nextcloud,
    description={Logiciel libre de site d'hébergement de fichiers et une plateforme de collaboration}
}
\newglossaryentry{NocoDB}
{
    name=NocoDB,
    description={Outils permettant le traitement en feuille de calcul de bases de données SQL}
}

% OOOOOOOOOOOOOOOOOOOOOOOOOOOOOOOOOOOOOOOOOOOOOOOOOOOOOOOOOOOOOOOOOOOOOOOOOOOOOOOOOOOOOOOOOOOOOOOOOOO
\newglossaryentry{openvas}
{
    name=Openvas,
    description={Scanner de vulnérabilité Nessus dont le but est de permettre un développement libre de l’outil qui est maintenant sous licence propriétaire}
}

\newglossaryentry{OpenSCAP}
{
    name=OpenSCAP,
    description={Outils pour aider les administrateurs et les auditeurs à évaluer, mesurer et appliquer les lignes de base de sécurité}
}

\newglossaryentry{OSD}
{
    name=OSD,
    description={Object Storage Deamon. Dans ceph, un OSD est le deamon responsable du suivi des stockages sous forme d'objet}
}

\newglossaryentry{OCSInventory}
{
    name=OCSInventory,
    description={Open Computer and Software Inventory. Application permettant de réaliser un inventaire sur la configuration matérielle des machines du réseau, sur les logiciels qui y sont installés et de visualiser ces informations grâce à une interface web}
}

% PPPPPPPPPPPPPPPPPPPPPPPPPPPPPPPPPPPPPPPPPPPPPPPPPPPPPPPPPPPPPPPPPPPPPPPPPPPPPPPPPPPPPPPPPPPPPPPPPPP
\newglossaryentry{PWA}
{
    name=PWA,
    description={Une progressive web app est une application web qui consiste en des pages ou des sites web, et qui peuvent apparaître à l'utilisateur de la même manière que les applications natives ou les applications mobiles}
}

\newglossaryentry{PAS}
{
    name=PAS,
    description={Plan d'Assurance Sécurité a pour but de préciser comment les prestataires se conforment aux exigences de cybersécurité}
}

\newglossaryentry{plugin}
{
    name=Plugin,
    description={En informatique, un plugin ou plug-in, aussi nommé module d'extension, module externe, greffon, plugiciel, ainsi qu'add-in ou add-on en France, est un logiciel conçu pour être greffé à un autre logiciel à travers une interface prévue à cet effet, et apporter à ce dernier de nouvelles fonctionnalités}
}

\newglossaryentry{passhport}
{
    name=Passhport,
    description={Solution de gestion et de sécurisation des accès SSH. PaSSHport est un bastion SSH protégeant les accès sécurisés des systèmes d'information}
}

\newglossaryentry{playbooks}
{
    name=Playbook,
    description={Les Playbooks Ansible offrent un système de gestion de configuration et de déploiement multi-machine reproductible, réutilisable et simple, bien adapté au déploiement d'applications complexes}
}

\newglossaryentry{plateforme web}
{
    name=plateforme web,
    description={Une plateforme web est un ensemble de services web. Le contenu des plate-formes web provient en grande partie des utilisateurs qui peuvent diffuser et partager des contenus de nature textuelle ou multimédia}
}

\newglossaryentry{Peertube}
{
    name=Peertube,
    description={Logiciel libre d'hébergement de vidéo décentralisé permettant la diffusion en pair à pair, et un média social sur lequel les utilisateurs peuvent envoyer, regarder, commenter, évaluer et partager des vidéos en streaming}
}

\newglossaryentry{POC}
{
    name=POC,
    description={Proof of Concept ou Preuve de concept. Démonstration de faisabilité, c'est à dire une réalisation expérimentale concrète et préliminaire, courte ou incomplète, illustrant une certaine méthode ou idée afin d'en démontrer ou pas la faisabilité}
}

\newglossaryentry{PSSI}
{
    name=PSSI,
    description={Politique de Sécurité du Système d'Information. Plan d'actions définies pour maintenir un certain niveau de sécurité. Elle reflète la vision stratégique de la direction de l'organisme en matière de sécurité des systèmes d'information}
}

\newglossaryentry{PFE}
{
    name=PFE,
    description={Projet de Fin d'Etude. Désigne à l'ESEO à un projet d'une durée d'un semestre pouvant être effectué en entreprise ou non}
}

\newglossaryentry{Proxmox}
{
    name=Proxmox,
    description={Solution de virtualisation libre basée sur l'hyperviseur Linux KVM, et offre aussi une solution de containers avec LXC}
}

\newglossaryentry{proxy}
{
    name=Proxy,
    description={Composant logiciel informatique qui joue le rôle d'intermédiaire en se plaçant entre deux hôtes pour faciliter ou surveiller leurs échanges}
}
\newglossaryentry{Passhport}
{
    name=Passhport,
    description={Solution de gestion et de sécurisation des accès SSH}
}
\newglossaryentry{PHP}
{
    name=PHP,
    description={Hypertext Preprocessor. Langage de programmation principalement utilisé pour produire des pages Web dynamiques via un serveur HTTP, mais pouvant également fonctionner comme n'importe quel langage interprété de façon locale. PHP est un langage impératif orienté objet}
}


% QQQQQQQQQQQQQQQQQQQQQQQQQQQQQQQQQQQQQQQQQQQQQQQQQQQQQQQQQQQQQQQQQQQQQQQQQQQQQQQQQQQQQQQQQQQQQQQQQQQ

% RRRRRRRRRRRRRRRRRRRRRRRRRRRRRRRRRRRRRRRRRRRRRRRRRRRRRRRRRRRRRRRRRRRRRRRRRRRRRRRRRRRRRRRRRRRRRRRRRRR
\newglossaryentry{RSE}
{
    name=RSE,
    description={Responsabilité Sociétale des Entreprises. La responsabilité sociétale des entreprises désigne la prise en compte par les entreprises, sur une base volontaire, et parfois juridique, des enjeux environnementaux, sociaux, économiques et éthiques dans leurs activités.}
}

\newglossaryentry{Rudder}
{
    name=Rudder,
    description={Logiciel libre de configuration automatique de serveurs. Il se veut simple d'utilisation, orienté web et applique un raisonnement dirigé par les rôles. Il s'appuie sur des agents légers installés localement sur chaque machine gérée}
}

\newglossaryentry{Rocky}
{
    name=Rocky Linux,
    description={Distribution Linux. Basée sur le code source du système d'exploitation Red Hat Enterprise Linux.}
}

\newglossaryentry{RGPD}
{
    name=RGPD,
    description={Règlement Générale sur la Protection des Données}
}
\newglossaryentry{Redmine}
{
    name=Redmine,
    description={Application web libre de gestion de projets, développée en Ruby sur la base du framework Ruby on Rails}
}
\newglossaryentry{RBAC}
{
    name=RBAC,
    description={Role-Based Access Control. Le contrôle d'accès basé sur les rôles est un modèle de contrôle d'accès à un système d'information dans lequel chaque décision d'accès est basée sur le rôle auquel l'utilisateur est associé}
}


% SSSSSSSSSSSSSSSSSSSSSSSSSSSSSSSSSSSSSSSSSSSSSSSSSSSSSSSSSSSSSSSSSSSSSSSSSSSSSSSSSSSSSSSSSSSSSSSSSSS
\newglossaryentry{Slack}
{
    name=Slack,
    description={Slack est une plateforme de communication collaborative propriétaire ainsi qu'un logiciel de gestion de projets}
}
\newglossaryentry{SAS}
{
    name=SAS,
    description={Société par Actions Simplifiée}
}

\newglossaryentry{SARL}
{
    name=SARL,
    description={Société à Responsabilité Limitée. Société de capitaux à risque limité répandue dans les pays de tradition civiliste}
}

\newglossaryentry{SA}
{
    name=SA,
    description={Société Anonyme. Société commerciale où la responsabilité est limitée jusqu'à concurrence des apports, et qui présente des caractéristiques d'une société de personnes (2 à 100 personnes), notamment parce que les parts détenues dans le capital ne sont pas librement accessibles sans accord de tout ou partie des associés. }
}

\newglossaryentry{SCAP}
{
    name=SCAP,
    description={Security Content Automation Protocol. Ses spécifications visent à faciliter et automatiser les échanges d'informations entre les outils de sécurité pour tous les types d'équipements connectés au réseau : inventaire, vulnérabilités, éléments de configuration et de durcissement}
}

\newglossaryentry{SSD}
{
    name=SSD,
    description={Solid State Drive. Matériel informatique peremttant le stockage de données}
}

\newglossaryentry{sysadmin}
{
    name=Sysadmin,
    description={Le pôle sysadmin est en charge du développement, du suivi et de la maintenance du SI de l'entreprise. C'est l'équivalent d'administrateurs systèmes}
}

\newglossaryentry{SQL}
{
    name=SQL,
    description={Langage informatique normalisé servant à exploiter des bases de données relationnelles. La partie langage de manipulation des données de SQL permet de rechercher, d'ajouter, de modifier ou de supprimer des données dans les bases de données relationnelles}
}

\newglossaryentry{SCOP}
{
    name=SCOP,
    description={Société COopérative et Participative}
}

\newglossaryentry{SI}
{
    name=SI,
    description={Système d'information. Ensemble organisé de ressources qui permet de collecter, stocker, traiter et distribuer de l'information, en général grâce à un réseau d'ordinateurs}
}

\newglossaryentry{SSH}
{
    name=SSH,
    description={Secure SHell. Protocole de communication sécurisée basé sur un échange de clé privé et publique}
}

\newglossaryentry{SSL}
{
    name=SSL,
    description={Secure Sockets Layer. Protocole cryptographique le plus largement utilisé pour assurer la sécurité des communications sur Internet}
}
\newglossaryentry{SNMP}
{
    name=SNMP,
    description={Simple Network Management Protocol ou Protocole simple de gestion de réseau. Protocole de communication qui permet aux administrateurs réseau de gérer les équipements du réseau, de superviser et de diagnostiquer des problèmes réseaux et matériels à distance}
}

% TTTTTTTTTTTTTTTTTTTTTTTTTTTTTTTTTTTTTTTTTTTTTTTTTTTTTTTTTTTTTTTTTTTTTTTTTTTTTTTTTTTTTTTTTTTTTTTTTTT
\newglossaryentry{TLS}
{
    name=TLS,
    description={Transport Layer Security ou Sécurité de la couche de transport. Protocole de sécurisation des échanges par réseau informatique, notamment par Internet}
}
\newglossaryentry{Traefik}
{
    name=Traefik,
    description={Reverse-proxy et un load-balancer fait pour déployer principalement des conteneurs}
}
\newglossaryentry{Terraform}
{
    name=Terraform,
    description={Logiciel d'« infrastructure as code » publié en open-source par la société HashiCorp. Cet outil permet d'automatiser la construction des ressources d'une infrastructure de centre de données comme un réseau, des machines virtuelles, un groupe de sécurité ou une base de données}
}

% UUUUUUUUUUUUUUUUUUUUUUUUUUUUUUUUUUUUUUUUUUUUUUUUUUUUUUUUUUUUUUUUUUUUUUUUUUUUUUUUUUUUUUUUUUUUUUUUUUU

% VVVVVVVVVVVVVVVVVVVVVVVVVVVVVVVVVVVVVVVVVVVVVVVVVVVVVVVVVVVVVVVVVVVVVVVVVVVVVVVVVVVVVVVVVVVVVVVVVVV
\newglossaryentry{VIP}
{
    name=VIP,
    description={Virtual IP. Adresse IP virtuelle non connectée à un ordinateur ou une carte réseau spécifiques. Les paquets entrants sont envoyés à l'adresse IP virtuelle, mais en réalité ils circulent tous via des interfaces réseau réelles}
}

\newglossaryentry{vhosts}
{
    name=Vhosts,
    description={Virutal Host. Méthode que les serveurs tels que serveurs Web utilisent pour accueillir plus d'un nom de domaine sur le même ordinateur, parfois sur la même adresse IP, tout en maintenant une gestion séparée de chacun de ces noms}
}

% WWWWWWWWWWWWWWWWWWWWWWWWWWWWWWWWWWWWWWWWWWWWWWWWWWWWWWWWWWWWWWWWWWWWWWWWWWWWWWWWWWWWWWWWWWWWWWWWWWW
\newglossaryentry{Wifi}
{
    name=Wifi,
    description={Protocole IEEE 802.11. Ensemble de normes concernant les réseaux sans fil locaux}
}

\newglossaryentry{webinaire}
{
    name=Webinaire,
    description={Mot-valise associant les mots Web et séminaire, créé pour désigner toutes les formes de réunions interactives de type séminaire faites par Internet généralement dans un but de travail collaboratif ou de transmission d'informations pour une audience plus ou moins importante en nombre}
}


\newglossaryentry{Windows}
{
    name=Windows,
    description={Interface graphique unifiée produite par Microsoft, qui est devenue ensuite une gamme de systèmes d’exploitation à part entière, principalement destinés aux ordinateurs compatibles PC}
}
\newglossaryentry{Wazuh}
{
    name=Wazuh,
    description={Plateforme open source utilisée pour la prévention, la détection et la réponse aux menaces}
}

% XXXXXXXXXXXXXXXXXXXXXXXXXXXXXXXXXXXXXXXXXXXXXXXXXXXXXXXXXXXXXXXXXXXXXXXXXXXXXXXXXXXXXXXXXXXXXXXXXXX

% YYYYYYYYYYYYYYYYYYYYYYYYYYYYYYYYYYYYYYYYYYYYYYYYYYYYYYYYYYYYYYYYYYYYYYYYYYYYYYYYYYYYYYYYYYYYYYYYYYY
\newglossaryentry{YesWeHack}
{
    name=YesWeHack,
    description={Plateforme permettant la mise en contact entre des entreprises et des hackers éthiques}
}

% ZZZZZZZZZZZZZZZZZZZZZZZZZZZZZZZZZZZZZZZZZZZZZZZZZZZZZZZZZZZZZZZZZZZZZZZZZZZZZZZZZZZZZZZZZZZZZZZZZZZ
\newglossaryentry{Zammad}
{
    name=Zammad,
    description={Service d'assistance gratuit ou un système de suivi des problèmes}
}

































