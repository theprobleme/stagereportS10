\subsubsection{Mise en oeuvre}
\paragraph{L'utilisation de variable avec docker}
Afin d'expliquer concrètement la gestion des variables, on considère 2 variables :
\begin{itemize}
    \item \verb|NEXTCLOUD_IMAGE| : le nom de l'image Nextcloud utilisée qui n'est pas un secret.
    \item \verb|NEXTCLOUD_ADMIN_PASSWORD| : le mot de passe de l'utilisateur administrateur de Nextcloud qui est un secret.
\end{itemize}

Dans un script docker-compose, afin d'utiliser les variables d'un fichier \verb|.env|, on applique la syntaxe suivant :
\begin{verbatim}
services:
  app:
     image: "${NEXTCLOUD_IMAGE}:${NEXTCLOUD_VERSION}"
     environment:
       - NEXTCLOUD_ADMIN_PASSWORD=${NEXTCLOUD_ADMIN_PASSWORD}
\end{verbatim}
Cette syntaxe signifie que la valeur des paramètres \verb|NEXTCLOUD_IMAGE| et \verb|NEXTCLOUD_ADMIN_PASSWORD| seront prises dans le fichier \verb|.env|.

\paragraph{Le templating avec Jinja}
Jinja est un moteur de templates pour le langage de programmation Python. Dans notre cas, Jinja est utilisé pour faire le template du fichier \verb|.env| pour que celui-ci utilise les variables d'Ansible.

Ainsi, le fichier \verb|env_template_nextcloud.j2| contiendra :
\begin{verbatim}
## Nextcloud variables used in /apps/nextcloud/docker-compose.yml
NEXTCLOUD_IMAGE={{ nextcloud_docker_image }}
# Variables to change in the Vault file vault.yaml
## Nextcloud
NEXTCLOUD_ADMIN_PASSWORD={{ vault_nextcloud.admin_password }}
\end{verbatim}